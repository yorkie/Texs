% abstract.tex

\noindent
计算机技术的不断发展,互联网技术不断崛起,Web逐渐成为了人们日常生活中不可获取的生命要素。报刊杂志等纸媒作为上一个时代的有利传播工具,将其Web化的步伐已刻不容缓。利用Web快速、精准以及多元化等特性,可以实现一个分众(Focus)、精准(Precise)、互动(Interactive)的多元化平台媒体。

\begin{description}
 	\item[分众] 即用户差异化,服务方有效地区分受众人群
	\item[精准] 即准确把握目标受众的商业需求
	\item[互动] 即能够实现产业链上下游(上游:厂商客户;下游:终端用户)之间的信息互动、相互沟通、互通了解需求,形成有价值的互动信息回馈链条
\end{description}

\indent
而目前亟待解决的问题来自于在思想以及形式上如何从传统媒体平台转换到新媒体平台,现在的解决方案大多分为两类,一类是以电子媒体为主,多以Blog为主,主要战斗力集中在新媒体中与竞争对手进行周旋,在有余力之下,定期地推出某个系列的纸质/电子出版物,其中以SmashingMagzine/iFanr为代表。而另一类则以各大地方报纸、传统媒体行业为主,与前者相反,这些出版社的注意力仍然集中在纸媒体,而在新媒体平台缺乏创新力、创造力的做法,很少能够换得其用户的满堂喝彩。

\indent
本文将会以上述这一问题为根本需求依据,结合在实践中的真实经验,介绍了如何通过Express框架等一系列以ECMAScript为核心的技术体系,进行快速搭建高效、稳定、可迭代的Web应用以及其开发模型。

\noindent
论文共分为七个章节:

\begin{description}
	\item[第一章] 对摘要中的问题进行细致的分析,从企业的角度出发,解决其根本问题
	\item[第二章] 介绍开发平台,包含程序、测试工具、部署平台所使用到的一系列开源工具
	\item[第三章] 从前端出发,说明其架构思路,并提供前端架构图
	\item[第四章] 从后端出发,说明其架构思路,并提供后端架构图
	\item[第五章] 系统的各个组件的设计/实现思路
	\item[第六章] 单独就API设计、缓存以及部署等重要环节,与具体业务场景结合起来一一说明
	\item[第七章] 性能测试与分析
	\item[第八章] 总结该系统的总体设计思路以及发展路线
 	\item[关键字] 数字报刊;NodeJS;MongoDB;系统设计
\end{description}