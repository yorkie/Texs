% Chapter1 开发平台

\subsection{开发语言}
\indent
本系统采用了一些相对比较新潮的开发语言,使用了SeaJS+NodeJS+MongoDB的作为前后端架构,其优点是快速搭建,对前后端整合更有利。

\subsection{NodeJS的前世今生}
\indent
NodeJS是一个可以快速构建网络服务及应用的平台。该平台的构建是基于Chrome's JavaScript runtime,也就是说,实际上它是对GoogleV8引擎进行了封装。

V8引 擎执行Javascript的速度非常快,性能非常好。Node对一些特殊用例进行了优化,提供了替代的API,使得V8在非浏览器环境下运行得更好。

V8引擎本身使用了一些最新的编译技术。这使得用Javascript这类脚本语言编写出来的代码与用C这类高级语言写出来的代码性能相差无几,却节省了开发成本。对性能的苛求是Node的一个关键因素。 Javascript是一个事件驱动语言,Node利用了这个优点,编写出可扩展性高的服务器。Node采用了一个称为“事件循环(event loop)”的架构,使得编写可扩展性高的服务器变得既容易又安全。提高服务器性能的技巧有多种多样。Node选择了一种既能提高性能,又能减低开发复杂度的架构。这是一个非常重要的特性。并发编程通常很复杂且布满地雷。Node绕过了这些,但仍提供很好的性能。

Node采用一系列“非阻塞”库来支持事件循环的方式。本质上就是为文件系统、数据库之类的资源提供接口。向文件系统发送一个请求时,无需等待硬盘(寻址并检索文件),硬盘准备好的时候非阻塞接口会通知Node。该模型以可扩展的方式简化了对慢资源的访问, 直观,易懂。尤其是对于熟悉onmouseover、onclick等DOM事件的用户,更有一种似曾相识的感觉。

虽然让Javascript运行于服务器端不是Node的独特之处,但却是其一强大功能。不得不承认,浏览器环境限制了我们选择编程语言的自由。任何服务器与日益复杂的浏览器客户端应用程序间共享代码的愿望只能通过Javascript来实现。虽然还存在其他一些支持Javascript在服务器端 运行的平台,但因为上述特性,Node发展迅猛,成为事实上的平台。

在Node启动的很短时间内,社区就已经贡献了大量的扩展库(模块)。其中很多是连接数据库或是其他软件的驱动,但还有很多是凭他们的实力制作出来的非常有用的软件。

最后,不得不提到的是Node社区。虽然Node项目还非常年轻,但很少看到对一个项目如此狂热的社区。不管是新手,还是专家,大家都围绕着项目,使用并贡献自己的能力,致力于打造一个探索、支持、分享、听取建议的乐土。

\subsection{Express.js——NodeJS平台上的Web框架}
\indent
Express.js是基于NodeJS,高性能、一流的web开发框架,它通过对NodeJS的部分接口进行重新封装,使其更易于适用在web的世界中,并且提供了良好的插件机制,可以灵活地向应用中添加模版引擎,中间件等,因而使其具有一定的生态活力。

\subsection{NoSQL与MongoDB}
\indent
随着互联网web2.0网站的兴起,非关系型的数据库成了一个极其热门的新领域,非关系数据库产品的发展非常迅速。而传统的关系数据库在应付web2.0网站,特别是超大规模和高并发的SNS类型的web2.0纯动态网站已经显得力不从心,暴露了很多难以克服的问题,例如:

\begin{description}
    \item[对数据库高并发读写的需求] web2.0网站要根据用户个性化信息来实时生成动态页面和提供动态信息,所以基本上无法使用动态页面静态化技术,因此数据库并发负载非常高,往往要达到每秒上万次读写请求。关系数据库应付上万次SQL查询还勉强顶得住,但是应付上万次SQL写数据请求,硬盘IO就已经无法承受了。其实对于普通的BBS网站,往往也存在对高并发写请求的需求。
    \item[对海量数据的高效率存储和访问的需求] 对于大型的SNS网站,每天用户产生海量的用户动态,以国外的Friendfeed为例,一个月就达到了2.5亿条用户动态,对于关系数据库来说,在一张2.5亿条记录的表里面进行SQL查询,效率是极其低下乃至不可忍受的。再例如大型web网站的用户登录系统,例如腾讯,盛大,动辄数以亿计的帐号,关系数据库也很难应付。
    \item[对数据库的高可扩展性和高可用性的需求] 在基于web的架构当中,数据库是最难进行横向扩展的,当一个应用系统的用户量和访问量与日俱增的时候,你的数据库却没有办法像web server和app server那样简单的通过添加更多的硬件和服务节点来扩展性能和负载能力。对于很多需要提供24小时不间断服务的网站来说,对数据库系统进行升级和扩展是非常痛苦的事情,往往需要停机维护和数据迁移,为什么数据库不能通过不断的添加服务器节点来实现扩展呢?
\end{description}

\noindent
在上面提到的“三高”需求面前,关系数据库遇到了难以克服的障碍,而对于web2.0网站来说,关系数据库的很多主要特性却往往无用武之地,例如:

\begin{description}
    \item[数据库事务一致性需求] 很多web实时系统并不要求严格的数据库事务,对读一致性的要求很低,有些场合对写一致性要求也不高。因此数据库事务管理成了数据库高负载下一个沉重的负担。
    \item[数据库的写实时性和读实时性需求] 对关系数据库来说,插入一条数据之后立刻查询,是肯定可以读出来这条数据的,但是对于很多web应用来说,并不要求这么高的实时性。
    \item[对复杂的SQL查询,特别是多表关联查询的需求] 任何大数据量的web系统,都非常忌讳多个大表的关联查询,以及复杂的数据分析类型的复杂SQL报表查询,特别是SNS类型的网站,从需求以及产品设计角度,就避免了这种情况的产生。往往更多的只是单表的主键查询,以及单表的简单条件分页查询,SQL的功能被极大的弱化了。
\end{description}

\noindent
因此,关系数据库在这些越来越多的应用场景下显得不那么合适了,为了解决这类问题的非关系数据库应运而生。

NoSQL 是非关系型数据存储的广义定义。它打破了长久以来关系型数据库与ACID理论大一统的局面。NoSQL 数据存储不需要固定的表结构,通常也不存在连接操作。在大数据存取上具备关系型数据库无法比拟的性能优势。该术语在 2009 年初得到了广泛认同。

\paragraph{MongoDB}
是一个介于关系数据库和非关系数据库之间的产品,是非关系数据库当中功能最丰富,最像关系数据库的。他支持的数据结构非常松散,是类似json的bjson格式,因此可以存储比较复杂的数据类型。Mongo最大的特点是他支持的查询语言非常强大,其语法有点类似于面向对象的查询语言,几乎可以实现类似关系数据库单表查询的绝大部分功能,而且还支持对数据建立索引。它的特点是高性能、易部署、易使用,存储数据非常方便。


\subsection{SeaJS前端平台}
\indent
SeaJS是一个遵循CommonJS规范的JavaScript模块加载框架,可以实现其的模块化开发及加载机制。与jQuery等框架不同,SeaJS不会扩展封装语言特性,而只是实现JavaScript的模块化及按模块加载。SeaJS的主要目的是令JavaScript开发模块化并可以轻松愉悦进行加载,将前端工程师从繁重的JavaScript文件及对象依赖处理中解放出来,可以专注于代码本身的逻辑。SeaJS可以与jQuery这类框架完美集成。使用SeaJS可以提高JavaScript代码的可读性和清晰度,解决目前JavaScript编程中普遍存在的依赖关系混乱和代码纠缠等问题,方便代码的编写和维护。\\[0.1cm]
\indent
SeaJS的作者是淘宝前端工程师玉伯。\\
\indent
SeaJS本身遵循KISS(Keep It Simple, Stupid)理念进行开发,其本身仅有个位数的API,因此学习起来毫无压力。在学习SeaJS的过程中,处处能感受到KISS原则的精髓——仅做一件事,做好一件事。