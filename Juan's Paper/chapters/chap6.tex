% Chapter1 核心功能实现

\subsection{缓存系统设计}
缓存系统是为了解决数据库服务器和web服务器之间的瓶颈。如果流量很大,这个瓶颈将会非常明显,每次数据库查询耗费的时间将会非常庞大。对于更新速度不是很快的网站,我们可以用静态化来避免过多的数据库查询,对于更新速度以秒计的网站,静态化也不会太理想,这时就可以用缓存系统来构建。

由于系统本身对于缓存需求不大,因此缓存系统组件采用了一个轻量级的in-memory数据存储组件——flashDB。

通过使用flashDB,系统在第一次启动服务器后,会把部分数据写入内存保存起来,与其他的缓存系统如:memcached或redis相比,flashDB并不会自己清空缓存在内存中的数据,这样做的原因之一是减少内部代码的复杂度,二是flashDB的应用场景都是针对一些轻存储容量的应用,因此对于空间的要求并不高。

\subsection{部署设计}
\indent
Occupation


\clearpage