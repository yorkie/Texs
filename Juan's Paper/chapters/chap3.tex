% Chapter 服务端架构细节

\subsection{NodeJS+MongoDB架构优势}
\indent
Occupation

\subsection{模块对象设计}
\indent
由于使用的是基于ECMAScript262的NodeJS为基本开发语言,因此并没有传统面向对象中的类,但并不能说其不支持面向对象的特性。我基于Express.js开发了一个数据库连接工具——express-model,另外它也是一个建模工具,系统中所有的模型都被放在根目录下一个叫models的文件夹中,下面分别对其中的文件进行描述。

\subsubsection{express-model}
在描述对象模型之前,先来看看如何使用express-model,首先是更加仔细地对它进行一段描述:它为Express框架提供Model支持,可集成多种数据库(SQL \& NoSQL),灵活性高的轻量级工具。 由于Express官方并未显式地提供Model支持,express-model不仅能有效嵌入Express Mvc中,还支持多种需要Model的场景。

\clearpage

\noindent
\texttt{\large 创建一个Model:}

% create a model in express-model

\lstset{language=C}

\begin{lstlisting}[frame=single]
Models.define('yourModelName', function(exports) {
	exports.name = 'express-model 1.0'
	exports.getName = function() {
		return this.name
	}
})
\end{lstlisting}
~

\noindent
\texttt{\large 使用Model:}

% use a model in express-model

\lstset{language=C}

\begin{lstlisting}[frame=single]
var model = Models.use('yourModelName', function() {
	this.someVariable_1 = 1;
	this.someVariable_2 = 2;
})
response.render('yourViewName', model)
\end{lstlisting}

\subsubsection{user.js}

\noindent
\texttt{\large 其模型定义代码如下:}

% use a model in express-model

\lstset{language=C}

\begin{lstlisting}[frame=single]
var UserSchema = new Db.Schema({
	name: String,
	email: String,
	password: String,
	role: String
})
UserSchema.index({ name: 1, role: 1 })
\end{lstlisting}

\subsubsection{groups.js}

\noindent
\texttt{\large 其模型定义代码如下:}

% use a model in express-model

\lstset{language=C}

\begin{lstlisting}[frame=single]
var GroupSchema = new Db.Schema({
	content: [ String ]
}, {
	collection: 'groups'
})
\end{lstlisting}

\subsubsection{kan.js}

\noindent
\texttt{\large 其模型定义代码如下:}

% use a model in express-model

\lstset{language=C}

\begin{lstlisting}[frame=single]
var KanSchema = new Db.Schema({
	user: String,
	name: String,
	group: String,
	tags: String,
	description: String,
	cover: String
})
KanSchema.index({name: 1, group: 1})
\end{lstlisting}

\subsubsection{issue.js}

\noindent
\texttt{\large 其模型定义代码如下:}

% use a model in express-model

\lstset{language=C}

\begin{lstlisting}[frame=single]
var IssueSchema = new Db.Schema({
	title: String,
	type: Boolean,
	content: String,
	date: Date,
	kanId: String,
})
IssueSchema.index({kan: 1})
\end{lstlisting}

\subsection{数据库设计}
\indent
Occupation

\subsection{服务器(组)选型}
\indent
Occupation

\subsection{服务端架构图}
\indent
Occupation