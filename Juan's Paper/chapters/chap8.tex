% Chapter8 总结
本电子报刊发行系统整体上采用了SeaJS+NodeJS+MongoDB系统架构,为了更好地复用其业务逻辑,使得整体系统结构具有强内聚,弱耦合,将整个系统分为了三层:视图表现层、业务逻辑层,数据持久层。各层分别运用了最新潮的开源框架技术:如ExpressJS、moogoose,同时,为了紧贴系统本身业务需求,还开发了2个开源工具:express-model与flashDB。

SeaJS是一个遵循CommonJS规范的JavaScript模块加载框架,可以实现其的模块化开发及加载机制。与jQuery等框架不同,SeaJS不会扩展封装语言特性,而只是实现JavaScript的模块化及按模块加载。SeaJS的主要目的是令JavaScript开发模块化并可以轻松愉悦进行加载,将前端工程师从繁重的JavaScript文件及对象依赖处理中解放出来,可以专注于代码本身的逻辑。SeaJS可以与jQuery这类框架完美集成。使用SeaJS可以提高JavaScript代码的可读性和清晰度,解决目前JavaScript编程中普遍存在的依赖关系混乱和代码纠缠等问题,方便代码的编写和维护。

Express.js是基于NodeJS,高性能、一流的web开发框架,它通过对NodeJS的部分接口进行重新封装,使其更易于适用在web的世界中,并且提供了良好的插件机制,可以灵活地向应用中添加模版引擎,中间件等,因而使其具有一定的生态活力。

express-model是基于express.js的对象建模工具,通过它可以轻松地连接到各种数据库服务器,并对它们进行操作,express-model的使用,解放了原来绝大部分位于控制器内的代码,在models内的对象中,包含有大量的业务逻辑与数据集合,而这些都是架构在书写对象模型时严谨的模式内的,在应用过程中,我不断优化其内部代码实现以及性能,并不断地简化其API,虽然相较于其他成熟的建模工具,相形见绌,但凭借着以上优点,也一定有其用武之地。

MongoDB是一个介于关系数据库和非关系数据库之间的产品,是非关系数据库当中功能最丰富,最像关系数据库的。他支持的数据结构非常松散,是类似json的bson格式,因此可以存储比较复杂的数据类型。Mongo最大的特点是他支持的查询语言非常强大,其语法有点类似于面向对象的查询语言,几乎可以实现类似关系数据库单表查询的绝大部分功能,而且还支持对数据建立索引。它的特点是高性能、易部署、易使用,存储数据非常方便。

这些开源技术的结合,极大地提高了整个系统的开发效率,同时也提供了系统的可测试性,与易于维护的方便性。作者通过研究各种技术的特点与实现,将其运用到电子报刊发行系统中,取得了一定的成果。不过这些使用的技术都还处于发展阶段,其版本号与相关文档也在不断地修改、升级,技术不断地被完善,很值得我们去进一步研究。最后,对于该电子报刊发行系统的拓展,也值得进一步研究,由于制作比较仓促,部分方面如数据统计平台、在线阅读平台移动化等重要的方面都未能顾及到,不过这些工作将会是什么有意义的。



\clearpage